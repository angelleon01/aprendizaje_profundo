\documentclass[paper=a4, fontsize=11pt]{scrartcl} % A4 paper and 11pt font size

% ---- Entrada y salida de texto -----

\usepackage[T1]{fontenc} % Use 8-bit encoding that has 256 glyphs
\usepackage[utf8]{inputenc}
\usepackage[a4paper, total={6in, 9in}]{geometry}
%\usepackage{fourier} % Use the Adobe Utopia font for the document - comment this line to return to the LaTeX default

% ---- Idioma --------

\usepackage[spanish, es-tabla]{babel} % Selecciona el español para palabras introducidas automáticamente, p.ej. "septiembre" en la fecha y especifica que se use la palabra Tabla en vez de Cuadro

% ---- Grafos -----

\usepackage{tikz}
\usetikzlibrary{positioning}
% ---- Para codigo -----

\usepackage{listings} % Para incluir código
\usepackage{xcolor} % Para definir colores
\usepackage{caption} % Para subtítulos en imágenes y tablas

% Definición de colores para el código
\definecolor{bggray}{rgb}{0.95,0.95,0.95} % Color de fondo para el código
\definecolor{mygreen}{rgb}{0,0.6,0} % Color para comentarios
\definecolor{myblue}{rgb}{0,0,0.8} % Color para palabras clave
\definecolor{mygray}{rgb}{0.5,0.5,0.5} % Color para números

% Configuración del paquete listings para el código
\lstset{
    language=Python, % Lenguaje del código
    backgroundcolor=\color{bggray}, % Fondo gris claro
    basicstyle=\ttfamily\small, % Estilo de fuente básico
    keywordstyle=\color{myblue}, % Color de palabras clave
    commentstyle=\color{mygreen}, % Color de comentarios
    stringstyle=\color{mygray}, % Color de strings
    frame=single, % Marco alrededor del código
    rulecolor=\color{black}, % Color del marco
    showstringspaces=false, % No mostrar espacios en los strings
    breaklines=true, % Permitir salto de línea en código largo
    numbers=left, % Numeración a la izquierda
    numberstyle=\tiny\color{mygray}, % Color y tamaño de los números
}


% ---- Otros paquetes ----


\usepackage{amsmath,amsfonts,amsthm} % Math packages
%\usepackage{graphics,graphicx, floatrow} %para incluir imágenes y notas en las imágenes
\usepackage{graphics,graphicx, float, url} %para incluir imágenes y colocarlas
\usepackage{eurosym}
% Para hacer tablas comlejas
%\usepackage{multirow}
%\usepackage{threeparttable}

%\usepackage{sectsty} % Allows customizing section commands
%\allsectionsfont{\centering \normalfont\scshape} % Make all sections centered, the default font and small caps

%Esto es para hipervinculos
\usepackage[hidelinks]{hyperref}



\usepackage{tikz}
\usetikzlibrary{positioning}
% para grafos


\hypersetup{
    colorlinks=true,
    linkcolor=blue,
    filecolor=blue,      
    urlcolor=blue,
    citecolor=blue,
    pdftitle={Overleaf Example},
    pdfpagemode=FullScreen,
    }

\urlstyle{same}
\usepackage{gensymb}
\usepackage{fancyhdr} % Custom headers and footers
\pagestyle{fancyplain} % Makes all pages in the document conform to the custom headers and footers
\fancyhead{} % No page header - if you want one, create it in the same way as the footers below
\fancyfoot[L]{} % Empty left footer
\fancyfoot[C]{} % Empty center footer
\fancyfoot[R]{\thepage} % Page numbering for right footer
\renewcommand{\headrulewidth}{0pt} % Remove header underlines
\renewcommand{\footrulewidth}{0pt} % Remove footer underlines
\setlength{\headheight}{13.6pt} % Customize the height of the header

\numberwithin{equation}{section} % Number equations within sections (i.e. 1.1, 1.2, 2.1, 2.2 instead of 1, 2, 3, 4)
\numberwithin{figure}{section} % Number figures within sections (i.e. 1.1, 1.2, 2.1, 2.2 instead of 1, 2, 3, 4)
\numberwithin{table}{section} % Number tables within sections (i.e. 1.1, 1.2, 2.1, 2.2 instead of 1, 2, 3, 4)

\setlength\parindent{0pt} % Removes all indentation from paragraphs - comment this line for an assignment with lots of text

\newcommand{\horrule}[1]{\rule{\linewidth}{#1}} % Create horizontal rule command with 1 argument of height

\title{	
\normalfont \normalsize 
\textsc{{\textbf{Aprendizaje Profundo (2024-2025)}} \\ Máster en Robótica e Inteligencia Artificial \\ Universidad de León} \\ [20pt] % Your university, school and/or department name(s)
\horrule{0.5pt} \\[0.4cm] % Thin top horizontal rule
\huge Práctica 2 \\   % The assignment title
\horrule{1.5pt} \\[0.2cm] % Thick bottom horizontal rule
}

\author{Sheila Martínez Gómez\\
Alejandro Mayorga Caro\\
Ángel Morales Romero\\
}
 % Nombre y apellidos
 % Incluye la fecha actual

\begin{document}

\maketitle
\newpage %inserta un salto de página

%\tableofcontents % para generar el índice de contenidos
%\pagebreak

\section{Cuestión 1} 

\begin{enumerate}
    \item Calcular el valor de la función \( W_0 \) en tres puntos \( x_0 = 0 \, e^0 \), \( x_1 = 1 \, e^1 \), \( x_2 = 2 \, e^2 \).
\end{enumerate}

\begin{gather*}
    w_0 = f^{-1}(x)\\
    w_0(x \cdot e^{x}) = x
\end{gather*}

\begin{gather*}
    w_0 = 0 \Rightarrow w_0(0 \cdot e^{0}) = 0\\ x=0,0000\\    y= 0,0000\\\\
    w_0 = 1 \Rightarrow w_0(1 \cdot e^{1}) = 1\\x=2,7183\\    y= 1,0000\\\\
    w_0 = 2 \Rightarrow w_0(2 \cdot e^{2}) = 1\\ x=14,7181\\    y= 2,0000\\\\  
X= \begin{bmatrix}
0,0000 \\
2,7183 \\
14,7781
\end{bmatrix}
Y= \begin{bmatrix}
0,0000 \\
1,0000 \\
2,0000
\end{bmatrix}
\end{gather*}


\newpage
\section{Cuestión 2}


\begin{enumerate}
    \item Entrenar la red neuronal de la figura 2 para la predicción de \( W_0 \) de acuerdo a los siguientes requisitos:
    \begin{itemize}
        \item La red dispone de una primera capa lineal con 3 neuronas, cuyos pesos iniciales son 
        \(\begin{pmatrix} 0.15 & -0.10 & 0.12 \end{pmatrix}\) y sus sesgos \(\begin{pmatrix} 0.3 & -0.20 & 0.07 \end{pmatrix}\).
        
        \item La siguiente capa consta de funciones de activación sigmoidal.
        
        \item La salida de la red procede de una siguiente capa lineal con 1 neurona, con pesos iniciales 
        \(\begin{pmatrix} 1.4 \\ 7.8 \\ 3.4 \end{pmatrix}\) y sesgo \(0.5\).
    \end{itemize}
    
    \item El entrenamiento se realiza utilizando como función de coste 
    \[
    C = \frac{1}{m} \sum_{i=0}^{m} (W_0(x_i) - \hat{y}_i)^2,
    \]
    siendo \( \hat{y}_i \) la predicción de la red para el dato de entrada \( x_i \), y \( m \) el número de datos de entrenamiento.
    
    \item El entrenamiento se realiza utilizando la estrategia de descenso por gradiente con momento, utilizando para la primera iteración momento nulo al no disponer de mejor valor.
    
    \item Ajustar los parámetros de la red 1 época utilizando la técnica de diferenciación automática en modo reverse con tasa de aprendizaje igual a \( 0.01 \).
\end{enumerate}

\subsection{Resolución Analítica del Problema}

\subsubsection{Definición de la red neuronal}

La red neuronal a estudiar consta de tres capas, con una primera capa lineal de 3 neuronas, una segunda capa con funciones de activación sigmoidal y una tercera capa lineal con una neurona. Los pesos y sesgos iniciales de la red son los siguientes:

\begin{itemize}
    \item Primera capa:
    \[
    W_u = \begin{pmatrix} 0.15 & -0.10 & 0.12 \end{pmatrix}, \quad b_u = \begin{pmatrix} 0.3 & -0.20 & 0.07 \end{pmatrix}.
    \]
    
    \item Capa de salida:
    \[
    W_y = \begin{pmatrix} 1.4 \\ 7.8 \\ 3.4 \end{pmatrix}, \quad b_y = 0.5.
    \]

\end{itemize}

\subsubsection{Función de Coste}

La función de coste utilizada para evaluar el rendimiento de la red neuronal es el error cuadrático medio (MSE), dado por:

\[
C = \frac{1}{m} \sum_{i=1}^m (W_o\cdot(X_i) - y_{i})^2,
\]

donde \( y_i \) es el valor real de la función \( W_0 \) en el punto \( X_i \) y \( m \) es el número de datos de entrenamiento.

\subsection{Preparación de los Datos}

Para realizar esta tarea se ha utilizado Python, con la librería \textit{numpy} para el manejo de matrices y vectores. 
Primero se definen los datos como variables para poder trabajar con ellos. Se define la función de coste y se inicializan los pesos y sesgos de la red.

\vspace{2mm}
\begin{lstlisting}

    X = np.array([[0 * e**0],
    [1 * e**1],
    [2 * e**2]])

Y = np.array([[0],
    [1],
    [2]])

Wu = np.array([[0.15, -0.10, 0.12]])
bu = np.array([[0.3, -0.2, 0.07]])

Wy = np.array([[1.4],
     [7.8],
     [3.4]])

by = np.array([[0.5]])

\end{lstlisting}
\vspace{2mm}

\subsection{Propagación hacia Adelante}

Se realiza la propagación hacia adelante de la red neuronal, calculando las salidas de cada capa y la predicción final de la red. 

\vspace{2mm}
\begin{lstlisting}
    V_3 = Wu
    V_2 = bu
    V_1 = Wy
    V0 = by
    V1 = X @ V_3 + np.ones((m, 1)) @ V_2
    V2 = sigmoid(V1)
    V3 = V2 @ V_1 + np.ones((m, 1)) @ V0
    V4 = (1 / m) * ((Y - V3).T @ (Y - V3))
    f = V4
\end{lstlisting}
\vspace{2mm}

\subsection{Propagación hacia Atrás}

Antes de realizar la propagación hacia atrás, se ajustarán los pesos y sesgos con un \textit{learning rate} de 0.01, para acto seguido realizar la propagación hacia atrás per se.

\vspace{2mm}
\begin{lstlisting}
    V_3 = Wu
    V_2 = bu
    V_1 = Wy
    V0 = by
    V1 = X @ V_3 + np.ones((m, 1)) @ V_2
    V2 = sigmoid(V1)
    V3 = V2 @ V_1 + np.ones((m, 1)) @ V0
    V4 = (1 / m) * ((Y - V3).T @ (Y - V3))
    f = V4
\end{lstlisting}
\vspace{2mm}

\subsection{Resultados}

Tras realizar el proceso completo, se obtienen los siguientes valores para el coste, pesos y sesgos de la red neuronal dada:

\begin{table}[H]
    \centering
    \renewcommand{\arraystretch}{1.5}
    \begin{tabular}{|c|c|c|}
    \hline
    \textbf{Parámetro} & \textbf{Inicial} & \textbf{Final} \\
    \hline
    Coste & 29.59559685 & 13.69094522 \\
    \hline
    Wu & 
    $\begin{pmatrix} 0.15 & -0.1 & 0.12 \end{pmatrix}$ &
    $\begin{pmatrix} 0.08159339 & -0.68622434 & -0.11742454 \end{pmatrix}$ \\
    \hline
    bu & 
    $\begin{pmatrix} 0.3 & -0.2 & 0.07 \end{pmatrix}$ &
    $\begin{pmatrix} 0.27116171 & -0.37917983 & -0.00759286 \end{pmatrix}$ \\
    \hline
    Wy & 
    $\begin{pmatrix} 1.4 \\ 7.8 \\ 3.4 \end{pmatrix}$ &
    $\begin{pmatrix} 1.32598667 \\ 7.76211242 \\ 3.33274683 \end{pmatrix}$ \\
    \hline
    by & 0.5 & 0.39333115 \\
    \hline
    \end{tabular}
\end{table}

\newpage
\section{Cuestión 3}

Utilizando la red anterior para el cálculo de los valores objetivo de \( V_D \) en tres puntos \( (V_{cc} = 3, V_{cc} = 6, V_{cc} = 9) \), entrenar la red neuronal de la figura 3 para la predicción de \( V_D \) en función de \( V_{cc} \), con requisitos:

\begin{itemize}
    \item La red dispone de una primera capa lineal con 3 neuronas, cuyos pesos iniciales son 
    \[
    \begin{pmatrix}
        0.05 & 0.15 & -0.20 
    \end{pmatrix}
    \]
    y sus sesgos 
    \[
    \begin{pmatrix}
        0.23 & -0.10 & 0.17 
    \end{pmatrix}.
    \]
    
    \item La siguiente capa consta de funciones de activación sigmoidal.
    
    \item La capa siguiente es lineal, con 2 neuronas, cuyos pesos iniciales son 
    \[
    \begin{pmatrix}
        0.8 & -0.6 \\
        0.7 & 0.9 \\
        0.5 & -0.6 
    \end{pmatrix}
    \]
    y sus sesgos 
    \[
    \begin{pmatrix}
        0.45 & -0.34 
    \end{pmatrix}.
    \]
    
    \item La siguiente capa consta de funciones de activación ReLU.
    
    \item La salida de la red procede de una siguiente capa lineal con 1 neurona, con pesos iniciales 
    \[
    \begin{pmatrix}
        0.8 \\
        0.5 
    \end{pmatrix}
    \]
    y sesgo \( 0.7 \).
    
    \item El entrenamiento se realiza utilizando como función de coste 
    \[
    C = \frac{1}{m} \sum_{i=0}^{2} (V_{D_i} - \hat{y}_i)^2,
    \]
    siendo \( \hat{y}_i \) la predicción de la red para el dato de entrada \( V_{cc_i} \), y \( m \) el número de datos de entrenamiento.
    
    \item El entrenamiento se realiza utilizando la estrategia de descenso por gradiente con momento, utilizando para la primera iteración momento nulo al no disponer de mejor valor.
    
    \item Ajustar los parámetros de la red 1 época utilizando la técnica de diferenciación automática en modo reverse.
    
    \item Valores típicos, para un diodo de germanio, podrían ser:
    \begin{itemize}
        \item \( I_0 = 10^{-12} \, \text{A} \),
        \item \( \eta = 1 \),
        \item \( V_T = 0.026 \, \text{V} \).
    \end{itemize}
    
    \item Considere como valor de la resistencia \( R = 100 \, \Omega \).
\end{itemize}



Para resolver el ejercicio empezamos por calcular los valores objetivo de $V_D$ en los puntos ($V_{CC} = 3$, $V_{CC} = 6$, $V_{CC} = 9$).


Para ello partiremos de la siguiente ecuación deducida en el enunciado:
\[
V_D = V_{cc} + RI_0 - \eta V_T W_0 \left( \frac{RI_0}{\eta V_T} e^{\frac{V_{cc} + RI_0}{\eta V_T}} \right),
\]

\subsection{Resolución Analítica del Voltaje $V_D$}

Definimos los parámetros:
\[
a = V_{cc} + RI_0, \quad b = \frac{R I_0}{\eta V_T}, \quad \text{y} \quad c = \frac{1}{\eta V_T}.
\]
La solución para $V_D$ en función de $V_{cc}$ es:
\[
V_D = a - \eta V_T W_0 \left( b e^{ac} \right).
\]

Para obtener las soluciones utilizaremos python, declarando las constantes definidas por el enunciado y calculando el resultado del voltaje $V_D$ en cada punto de $V_{CC}$.

\vspace{2mm}
\begin{lstlisting}
# Constantes
I0 = 1e-12
eta = 1
VT = 0.026 
R = 100
Vcc_values = [3, 6, 9]

# Calculo de VD
def calculate_VD_steps(Vcc):
    a = Vcc + R * I0  # Calculamos 'a'
    b = R * I0 / (eta * VT)  # Calculamos 'b'
    exponent = (Vcc + R * I0) / (eta * VT)  # Calculamos el exponent
    W_argument = b * np.exp(exponent)  # Calculamos el argumento de la funcion Lambert
    W_result = lambertw(W_argument).real  # La evaluamos
    VD = a - eta * VT * W_result  # Calculo final de VD
    return VD

# Calculamos el VD de cada Vcc
VD = {Vcc: calculate_VD_steps(Vcc) for Vcc in Vcc_values}
\end{lstlisting}
\vspace{2mm}

Obteniendo de este modo los siguientes valores:
\[
\begin{pmatrix} 3 & 6 & 9\end{pmatrix} \Rightarrow \begin{pmatrix} 0.6212 & 0.6423 & 0.6538 \end{pmatrix}
\]


\subsection{Modelado con Red Neuronal}

Para aproximar el valor de $V_D$ utilizando una red neuronal, empleamos una arquitectura de tres capas, como se puede ver en la figura 3 del enunciado. Comenzamos por declarar las entradas $X = \begin{bmatrix} V_{cc} \end{bmatrix}$ y los pesos y sesgos para cada capa como se detalla a continuación.

\subsubsection{Primera Capa}

La salida de la primera capa es:
\[
S = W_1 \times X + b_1,
\]
donde $W_1$ y $b_1$ son los pesos y el sesgo de la primera capa, definidos como:
\[
W_1 = \begin{bmatrix} 0.05 \\ 0.15 \\ -0.20 \end{bmatrix}, \quad b_1 = \begin{bmatrix} 0.23 \\ -0.10 \\ 0.17 \end{bmatrix}.
\]
Aplicamos la función de activación sigmoide a $S$, obteniendo:
\[
T = \sigma(S) = \frac{1}{1 + e^{-S}}.
\]

\subsubsection{Segunda Capa}

La segunda capa toma como entrada $T$ y produce una salida $U$ mediante la operación:
\[
U = W_2 \times T + b_2,
\]
donde $W_2$ y $b_2$ son los pesos y el sesgo de la segunda capa, dados por:
\[
W_2 = \begin{bmatrix} 0.8 & -0.6 & 0.5 \\ 0.7 & 0.9 & -0.6 \end{bmatrix}, \quad b_2 = \begin{bmatrix} 0.45 \\ -0.34 \end{bmatrix}.
\]
Aplicamos la función de activación ReLU a $U$ para obtener $V$:
\[
V = \text{ReLU}(U) = \max(0, U).
\]

\subsubsection{Capa de Salida}

Finalmente, la capa de salida produce el valor aproximado de $V_D$ como:
\[
Y_{pred} = W_3 \times V + b_3,
\]
donde $W_3 = \begin{bmatrix} 0.8 & 0.8 \end{bmatrix}$ y $b_3 = 0.7$. En esta capa, no aplicamos una función de activación adicional, obteniendo la salida directamente como $Y_{pred}$.

Declarando esto que hemos comentado en python queda de la siguiente manera:

\vspace{2mm}
\begin{lstlisting}
# Datos y objetivos
X = np.array([[3], [6], [9]])   # Entradas Vcc (3 filas x 1 columna)
Y = np.array([[VD[3]], [VD[6]], [VD[9]]])  # Objetivo VD
alpha = 1  # Tasa de aprendizaje

# Pesos y sesgos iniciales
# Primera capa
W1 = np.array([[0.05], [0.15], [-0.20]])  # (3x1)
b1 = np.array([[0.23], [-0.10], [0.17]])  # (3x1)

# Segunda capa
W2 = np.array([[0.8, -0.6, 0.5], [0.7, 0.9, -0.6]])  # (2x3)
b2 = np.array([[0.45], [-0.34]])  # (2x1)

# Capa de salida
W3 = np.array([[0.8, 0.8]])  # (1x2)
b3 = np.array([[0.7]])  # (1x1)

# Funciones de activacion
def sigmoid(x):
    return 1 / (1 + np.exp(-x))

def sigmoid_derivative(x):
    return sigmoid(x) * (1 - sigmoid(x))

def relu(x):
    return np.maximum(0, x)

def relu_derivative(x):
    return (x > 0).astype(float)
\end{lstlisting}
\vspace{10mm}
\begin{lstlisting}
# Propagacion hacia adelante
def forward(X):
    # Primera capa
    S = W1 @ X.T + b1  # Ajustamos para que las dimensiones sean compatibles
    T = sigmoid(S)

    # Segunda capa
    U = W2 @ T + b2
    V = relu(U)

    # Capa de salida
    Y_pred = W3 @ V + b3

    return S, T, U, V, Y_pred
\end{lstlisting}


\subsection{Función de Coste}

La función de coste utilizada para evaluar el rendimiento de la red neuronal es el error cuadrático medio (MSE), dado por:
\[
C = \frac{1}{m} \sum_{i=1}^m (V_D - Y_{pred})^2,
\]
donde $Y_{pred}$ es la predicción de $V_D$ para los voltajes $V_{CC}$ y $V_D$ es el valor real calculado anteriomente.

\vspace{2mm}
\begin{lstlisting}
# Coste (MSE)
def compute_cost(Y, Y_pred):
    m = Y.shape[1]
    cost = np.sum((Y - Y_pred) ** 2) / m
    return cost
\end{lstlisting}

\section{Propagación hacia Atrás}

Para ajustar los pesos y sesgos de la red, calculamos las derivadas parciales de la función de coste con respecto a cada parámetro, usando el método de retropropagación.

\subsubsection{Derivadas en la Capa de Salida}

\[
\frac{\partial C}{\partial Y_{pred}} = Y - Y_{pred},
\]
\[
\frac{\partial C}{\partial W_3} = \frac{1}{m} \left( \frac{\partial C}{\partial Y_{pred}} \right) V^T, \quad \frac{\partial C}{\partial b_3} = \frac{1}{m} \sum_{i=1}^m \frac{\partial C}{\partial Y_{pred}}.
\]

\subsubsection{Derivadas en la Segunda Capa}

\[
\frac{\partial C}{\partial U} = \left( W_3^T \frac{\partial C}{\partial Y_{pred}} \right) \odot \text{ReLU}'(U),
\]
\[
\frac{\partial C}{\partial W_2} = \frac{1}{m} \left( \frac{\partial C}{\partial U} \right) T^T, \quad \frac{\partial C}{\partial b_2} = \frac{1}{m} \sum_{i=1}^m \frac{\partial C}{\partial U}.
\]

\subsubsection{Derivadas en la Primera Capa}

\[
\frac{\partial C}{\partial S} = \left( W_2^T \frac{\partial C}{\partial U} \right) \odot \sigma'(S),
\]
\[
\frac{\partial C}{\partial W_1} = \frac{1}{m} \left( \frac{\partial C}{\partial S} \right) X^T, \quad \frac{\partial C}{\partial b_1} = \frac{1}{m} \sum_{i=1}^m \frac{\partial C}{\partial S}.
\]

\vspace{2mm}
\begin{lstlisting}
# Propagacion hacia atras (backpropagation)
def backward(X, Y, S, T, U, V, Y_pred):
    m = X.shape[1]

    # Derivada de la capa de salida
    dY_pred = Y_pred - Y  # dC/dY_pred
    dW3 = dY_pred @ V.T / m
    db3 = np.sum(dY_pred, axis=1, keepdims=True) / m

    # Derivadas de la segunda capa
    dV = W3.T @ dY_pred
    dU = dV * relu_derivative(U)
    dW2 = dU @ T.T / m
    db2 = np.sum(dU, axis=1, keepdims=True) / m

    # Derivadas de la primera capa
    dT = W2.T @ dU
    dS = dT * sigmoid_derivative(S)
    dW1 = dS @ X / m
    db1 = np.sum(dS, axis=1, keepdims=True) / m

    return dW1, db1, dW2, db2, dW3, db3
\end{lstlisting}
\vspace{2mm}

\subsection{Actualización de Parámetros}

Los pesos y sesgos se actualizan en cada iteración de entrenamiento con la regla de gradiente descendente:
\[
W_k := W_k - \alpha \frac{\partial C}{\partial W_k}, \quad b_k := b_k - \alpha \frac{\partial C}{\partial b_k},
\]
donde $\alpha$ es la tasa de aprendizaje (en este caso 1) y $k$ denota cada capa.

\vspace{2mm}
\begin{lstlisting}
# Actualizacion de pesos
def update_parameters(dW1, db1, dW2, db2, dW3, db3):
    global W1, b1, W2, b2, W3, b3
    W1 -= alpha * dW1
    b1 -= alpha * db1
    W2 -= alpha * dW2
    b2 -= alpha * db2
    W3 -= alpha * dW3
    b3 -= alpha * db3
\end{lstlisting}

\subsection{Conclusión}

Tras ejecutar el código expuesto anteriormente, obtenemos los siguientes valores:
\[
C = 1.0966
\]
\[
Y_{pred} = \begin{pmatrix} 1.6099 & 1.6879 & 1.7579 \end{pmatrix}
\]

Estos valores estan todavía un poco lejos de los valores reales, pero es la primera iteración.  Tras varias iteraciones estos valores deberían ir acercandose más a los valores reales.


\color{red}
\textbf{PROBAR EN DIFERENTES ITERACIONES Y EXPLICAR SI FUNCIONA O NO}



\end{document}
